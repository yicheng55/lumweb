\documentclass{beamer}
\usepackage{beamerthemeshadow}
\usepackage[utf8]{inputenc}
\usepackage{ngerman}
\begin{document}
\title{Lumweb}  
\author{Florian Hahn}
\date{\today} 

\begin{frame}
\titlepage
\end{frame}

\begin{frame}\frametitle{Inhaltsverzeichnis}\tableofcontents
\end{frame} 


\section{Probleme} 
\begin{frame}\frametitle{Probleme}
  bestehende Lösungen sind:\\
  \begin{itemize}
    \item unzureichend integriert
    \item unflexibel
  \end{itemize}
\end{frame}



\section{Entwicklung} 
\begin{frame}\frametitle{Entwicklung}
  \begin{description}
    \item [Team]
      \begin{itemize}
        \item Martin Anzinger
        \item Florian Hahn
      \end{itemize}
    \item[Entwicklungszeitraum] September 2009 - Mai 2010
  \end{description}
\end{frame}

\subsection{Ziel}
\begin{frame}\frametitle{Ziel}
  flexiblere Lösung für den industriellen Bereich durch:
  \begin{itemize}
    \item Standardprotokolle
       \begin{itemize}
         \item Ethernet
         \item Http
         \item Can-Bus
       \end{itemize}
    \item Touchscreen
  \end{itemize}
\end{frame}

\subsection{Aufbau}
\begin{frame}\frametitle{Aufbau}
  \begin{description}
    \item [Hardware] LM3S9B96 - 32 Bit ARM CPU
    \item [Software] 
      \begin{itemize}
        \item Betriebssystem: FreeRTOS
        \item Netzwerk: lwip Stack
      \end{itemize}
  \end{description}
\end{frame}

\subsection{Funktionsweise}
\begin{frame}\frametitle{Funktionsweise}
Funktionsweise
\end{frame}


\subsection{Vorteile}
\begin{frame}\frametitle{Vorteile}
  \begin{itemize}
    \item robuster
    \item sparsamer
    \item flexibler
  \end{itemize}
\end{frame}

\subsection{Demo}
\begin{frame}\frametitle{Demo}
  Es folgt eine Live Demonstration der Funktionsweise
\end{frame}

\end{document}